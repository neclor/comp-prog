% !TEX root = ./main.tex
%%%%%%%%%%%%%%%%%%%%%%%%%%%%%%%%%%%%%%%%%%%%%%%%%%%%%%%%%%%%%%%%%%%%%%%%%%%%%%%%%%%%%%%%%%
% Dans ce fichier, vous devez définir (Input/Output/O.U.) proprement et clairement le    %
% problème.
%
% Il est aussi demandé de réaliser une analyse complète (i.e., découpe en SPs)           %
%%%%%%%%%%%%%%%%%%%%%%%%%%%%%%%%%%%%%%%%%%%%%%%%%%%%%%%%%%%%%%%%%%%%%%%%%%%%%%%%%%%%%%%%%%

\section{Définition et Analyse du Problème}\label{analyse}
%%%%%%%%%%%%%%%%%%%%%%%%%%%%%%%%%%%%%%%%%%%%


\subsection{Input/Output}

\begin{itemize}
\item \textbf{Input}:
    Un tableau d'entiers $T$ de taille $N$,
    $N \geq 0$

\item \textbf{Output}:
    Un entier $k$ représentant la longueur du plus grand sous-tableau préfixe et suffixe
    $0$ si aucun sous-tableau non trivial ne satisfait la condition

\end{itemize}


\subsection{Découpe en sous-problèmes}

Nous décomposons le problème en deux Sp:
\begin{enumerate}
\item Recherche du plus grand prefixe-suffics de longeur $k$ possible de $N-1$ à $1$
\item Vérification que le préfixe et suffixe de longueur $k$ sont égaux
\end{enumerate}
