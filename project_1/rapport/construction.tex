% !TEX root = ./main.tex
%%%%%%%%%%%%%%%%%%%%%%%%%%%%%%%%%%%%%%%%%%%%%%%%%%%%%%%%%%%%%%%%%%%%%%%%%%%%%%%%%%%%%%%%%%
% Dans cette section, il est demandé d'appliquer l'approche constructive pour la         %
% construction de votre code.                                                            %
%                                                                                        %
% Pour chaque Sous-Problème (une sous-section/SP):                                       %
% - {Pré} INIT {INV}                                                                     %
% - déterminer le Critère d'Arrêt (et donc le Gardien de Boucle)                         %
% - {INV & B} ITER {INV}                                                                 %
% - {INV & !B} END {Post}                                                                %
% - Fonction de Terminaison (pensez à justifier sur base de l'Invariant Graphique)       %
% (une sous-sous-section/tiret)                                                          %
%%%%%%%%%%%%%%%%%%%%%%%%%%%%%%%%%%%%%%%%%%%%%%%%%%%%%%%%%%%%%%%%%%%%%%%%%%%%%%%%%%%%%%%%%%
\section{Approche Constructive}
%%%%%%%%%%%%%%%%%%%%%%%%%%%%%%%%

\begin{lstlisting}[caption={Un programme tout simple}]
int main(void)
{
    // Les commandes Latex sont permises dans les commentaires sur une ligne. Exemple : $x_i \leq a ^b$
    printf("Bonjour tout le monde !");
    /*
    Dans les commentaires sur plusieurs lignes, elles doivent être entourées
    de symboles définis par l'option « escapeinside » de \lstset
    (>\coms{$\sum_{i = 1}^N 1 = N$}<)
    La commande « \coms » permet de colorer correctement le code latex ajouté.
    Les accents et tous les autres diacritiques sont permis : àÀçÇéÉèÈêÊœŒ...
    */
    return 1; (>\label{code:ret}<)
}
\end{lstlisting}

Il est possible de faire référence à la ligne \ref{code:ret} de l'extrait de code.


