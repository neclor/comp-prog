% !TEX root = ./main.tex
%%%%%%%%%%%%%%%%%%%%%%%%%%%%%%%%%%%%%%%%%%%%%%%%%%%%%%%%%%%%%%%%%%%%%%%%%%%%%%%%%%%%%%%%%%
% Rédigez ici l'introduction de votre rapport.                                           %
%%%%%%%%%%%%%%%%%%%%%%%%%%%%%%%%%%%%%%%%%%%%%%%%%%%%%%%%%%%%%%%%%%%%%%%%%%%%%%%%%%%%%%%%%%
\section{Introduction}\label{introduction}
%%%%%%%%%%%%%%%%%%%%%%%



!!!!!! text generé par le AI!!!!!!!!!

Ce projet s'inscrit dans le cadre du cours INFO0947:
Compléments de Programmation à l'Université de Liège.
Nous devons résoudre le problème de recherche du plus grand
sous-tableau qui soit à la fois préfixe et suffixe d'un tableau donné, sans utiliser de tableau intermédiaire.

Le problème présente un intérêt particulier en algorithmique des chaînes et trouve des applications
dans divers domaines comme la bioinformatique ou le traitement de texte. Notre approche se base sur
une analyse systématique des propriétés des préfixes et suffixes, avec une contrainte forte sur
l'utilisation exclusive de boucles while et l'interdiction de structures de données auxiliaires.
