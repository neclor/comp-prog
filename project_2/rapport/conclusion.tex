% !TEX root = ./main.tex
%%%%%%%%%%%%%%%%%%%%%%%%%%%%%%%%%%%%%%%%%%%%%%%%%%%%%%%%%%%%%%%%%%%%%%%%%%%%%%%%%%%%%%%%%%
% Rédigez ici la conclusion de votre rapport.                                            %
%%%%%%%%%%%%%%%%%%%%%%%%%%%%%%%%%%%%%%%%%%%%%%%%%%%%%%%%%%%%%%%%%%%%%%%%%%%%%%%%%%%%%%%%%%
\section{Conclusion}\label{conclusion}
%%%%%%%%%%%%%%%%%%%%%
pour conclure ce rapport, nous pouvons dire que nous avons réussi à répondre au problème dans ça globalité.
En créant un programme capable de créer une course fictive et d'en utiliser tous les éléments qui la composent affin de trouver par quelles ville la course passe,
le meilleur temp qu'à mis un cycliste pour parcourir la distance entres deux villes, ou encore si la course forme un circuit. 
