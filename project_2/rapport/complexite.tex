% !TEX root = ./main.tex
%%%%%%%%%%%%%%%%%%%%%%%%%%%%%%%%%%%%%%%%%%%%%%%%%%%%%%%%%%%%%%%%%%%%%%%%%%%%%%%%%%%%%%%%%%
% Dans cette section, vous devez étudier complètement la complexité de votre code.       %
% Soyez le plus formel (i.e., mathématique) possible.                                    %
%%%%%%%%%%%%%%%%%%%%%%%%%%%%%%%%%%%%%%%%%%%%%%%%%%%%%%%%%%%%%%%%%%%%%%%%%%%%%%%%%%%%%%%%%%
\section{Complexité}\label{complexite}
%%%%%%%%%%%%%%%%%%%%

\subsection{Fonctions sur Escale}

\begin{itemize}
    \item escale\_create : $\mathcal{O}(1)$
    \item escale\_get\_name : $\mathcal{O}(1)$
    \item escale\_get\_x : $\mathcal{O}(1)$
    \item escale\_get\_y : $\mathcal{O}(1)$
    \item escale\_get\_best\_time : $\mathcal{O}(1)$
    \item escale\_set\_best\_time : $\mathcal{O}(1)$
    \item escale\_distance : $\mathcal{O}(1)$
    \item escale\_equal : $\mathcal{O}(1)$
\end{itemize}

\subsection{Opérations sur Course (tableau dynamique)}
Soit $n$ le nombre d'escales, $S$ la capacité du tableau.

\begin{align*}
&\texttt{course\_append} :
    \begin{cases}
        \mathcal{O}(1) & \text{(amorti)} \\
        \mathcal{O}(n) & \text{(réallocation)}
    \end{cases} \\
&\texttt{course\_pop} : \mathcal{O}(1) \\
&\texttt{course\_get\_escales\_count} : \mathcal{O}(1) \\
&\texttt{course\_total\_time} : \mathcal{O}(n) \\
&\texttt{course\_best\_time\_at}(i) : \mathcal{O}(1) \\
&\text{Espace mémoire} : \mathcal{O}(S),\ S \geq n
\end{align*}

\subsection{Opérations sur \texttt{Course} (liste chaînée)}

Soit $n$ le nombre d'escales.

\begin{align*}
&\texttt{course\_append} : \mathcal{O}(n) \\
&\texttt{course\_pop} : \mathcal{O}(n) \\
&\texttt{course\_get\_escales\_count} : \mathcal{O}(n) \\
&\texttt{course\_total\_time} : \mathcal{O}(n) \\
&\texttt{course\_best\_time\_at}(i) : \mathcal{O}(i) \\
&\texttt{course\_last} : \mathcal{O}(n) \\
&\text{Espace mémoire} : \mathcal{O}(n)
\end{align*}
