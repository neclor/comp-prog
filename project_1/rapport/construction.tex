% !TEX root = ./main.tex
%%%%%%%%%%%%%%%%%%%%%%%%%%%%%%%%%%%%%%%%%%%%%%%%%%%%%%%%%%%%%%%%%%%%%%%%%%%%%%%%%%%%%%%%%%
% Dans cette section, il est demandé d'appliquer l'approche constructive pour la         %
% construction de votre code.                                                            %
%                                                                                        %
% Pour chaque Sous-Problème (une sous-section/SP):                                       %
% - {Pré} INIT {INV}                                                                     %
% - déterminer le Critère d'Arrêt (et donc le Gardien de Boucle)                         %
% - {INV & B} ITER {INV}                                                                 %
% - {INV & !B} END {Post}                                                                %
% - Fonction de Terminaison (pensez à justifier sur base de l'Invariant Graphique)       %
% (une sous-sous-section/tiret)                                                          %
%%%%%%%%%%%%%%%%%%%%%%%%%%%%%%%%%%%%%%%%%%%%%%%%%%%%%%%%%%%%%%%%%%%%%%%%%%%%%%%%%%%%%%%%%%
\section{Approche Constructive}
%%%%%%%%%%%%%%%%%%%%%%%%%%%%%%%%

\begin{lstlisting}[caption={SP1}]
int prefixe_suffixe(int *T, const unsigned int N) {
    unsigned int k = N - 1;
    // {$T = T_0 \land N =N_0 \land k = N - 1$}
    while (k > 0) {
        // {$0 < k < N \land T = T_0 \land N =N_0$}
        if (pref_equal_suff(T, N, k)) return k;
        
        // {$T[0..k-1] \neq T[N-k..N-1]$}
        k--;
        // {$k = k - 1$}
    }
    // {$k = 0$}
    return 0;
    // {$T = T_0 \land N = N_0$}
}
\end{lstlisting}




\begin{lstlisting}[caption={SP2}]
static int pref_equal_suff(int *T, const unsigned int N, const unsigned int k) {

    unsigned int i = 0;
    // {$T = T_0 \land N =N_0 \land k = k_0$}
    while (i <= k - 1) {
        // {$T = T_0 \land N =N_0 \land k = k_0 \land 0 \leq i < k$}
        if (T[i] != T[N - k + i]) return 0;
        // {$T[i] = T[N - k + i]$}

        i++;
        //{$i = i + 1$}
    }
    //{$i = k$}
    return 1;
    // {$T = T_0 \land N =N_0 \land k = k_0 \land T[0..k-1] = T[N-k..N-1]$}
    }
\end{lstlisting}
