%%%%%%%%%%%%%%%%%%%%%%%%%%%%%%%%%%%%%%%%%%%%%%%%%%%%%%%%%%%%%%%%%%%%%%%%%%%%%%%%%%%%%%%%%%
% Ceci est le template à utiliser pour les milestones des projets d'INFO0947.            %
%                                                                                        %
% Vous devez décommenter et compléter les commandes introduites plus bas (intitule, ...) %
% avant de pouvoir compiler le fichier LaTeX.                                            %
%                                                                                        %
% Le PDF produit doit avoir la forme suivante: grpID_ProjetX_MilestoneY.pdf              %
% où:                                                                                    %
%   - grpID est votre identifiant de groupe (cfr. eCampus)                               %
%   - X est le numéro du projet (1 pour Construction, 2 pour TAD)                        %
%   - Y est le numéro du milestone (1 ou 2)                                              %
%%%%%%%%%%%%%%%%%%%%%%%%%%%%%%%%%%%%%%%%%%%%%%%%%%%%%%%%%%%%%%%%%%%%%%%%%%%%%%%%%%%%%%%%%%

% !TEX root = ./main.tex
% !TEX engine = latexmk -pdf
% !TEX buildOnSave = true
\documentclass[a4paper, 11pt, oneside]{article}

\usepackage[utf8]{inputenc}
\usepackage[T1]{fontenc}
\usepackage[french]{babel}
\usepackage{array}
\usepackage{shortvrb}
\usepackage{listings}
\usepackage[fleqn]{amsmath}
\usepackage{amsfonts}
\usepackage{fullpage}
\usepackage{enumerate}
\usepackage{graphicx}             % import, scale, and rotate graphics
\usepackage{subfigure}            % group figures
\usepackage{alltt}
\usepackage{url}
\usepackage{indentfirst}
\usepackage{eurosym}
\usepackage{listings}
\usepackage{color}
\usepackage[table,xcdraw,dvipsnames]{xcolor}

%%%%%%%%%%%%%%%%% TITRE %%%%%%%%%%%%%%%%
% Complétez et décommentez les définitions de macros suivantes :
\newcommand{\intitule}{Milestone 1}
\newcommand{\GrNbr}{33}
\newcommand{\PrenomUN}{Pavlov}
\newcommand{\NomUN}{Aleksandr}
\newcommand{\PrenomDEUX}{Gendebien}
\newcommand{\NomDEUX}{Alexandre}

%%%%%%%% ZONE PROTÉGÉE : MODIFIEZ UNE DES DIX PROCHAINES %%%%%%%%
%%%%%%%%            LIGNES POUR PERDRE 2 PTS.            %%%%%%%%
\title{INFO0947: \intitule}
\author{Groupe \GrNbr : \PrenomUN~\textsc{\NomUN}, \PrenomDEUX~\textsc{\NomDEUX}}
\date{}
\begin{document}

\maketitle
%%%%%%%%%%%%%%%%%%%% FIN DE LA ZONE PROTÉGÉE %%%%%%%%%%%%%%%%%%%%

%%%%%%%%%%%%%%%% MILESTONE %%%%%%%%%%%%%%%
% Écrivez le contenu de votre milestone ci-dessous.

\section{Production}
%%%%%%%%%%%%%%%%%%%%%

%%%%%%%%%%%%%%%%%%%%%%%%%%%%%%%%%%%%%%%%%%%%%%%%%%%%%%%%%%%%%%%%%%%%%%%%%%%%%%%%%%%%%%%%%%
% Dans cette section, vous rédigez la production demandée pour le milestone.             %
%                                                                                        %
% Soyez le plus précis et complet possible.  Toute production ne correspondant pas à ce  %
% qui est demandé sera considérée comme ``hors sujet'' et la séance de feedback sera     %
% contreproductive.                                                                      %
%%%%%%%%%%%%%%%%%%%%%%%%%%%%%%%%%%%%%%%%%%%%%%%%%%%%%%%%%%%%%%%%%%%%%%%%%%%%%%%%%%%%%%%%%%

En gros il faut crée une course a vélo,
une course est composé de plusieurs étapes(qui sont représenter par des coordonées)
on peut donc en déduire la distence entre deux viles (ou étapes)

\subsection{Escale}

\textbf{Type:}
\begin{itemize}
    \item[] Escale
\end{itemize}

\textbf{Utilise:}
\begin{itemize}
    \item[] Float, String
\end{itemize}

\textbf{Opérations:}
\begin{itemize}
    \item[] create: String $\times$ Float $\times$ Float $\to$ Escale
    \item[] getName: Escale $\to$ String
    \item[] getX: Escale $\to$ Float
    \item[] getY: Escale $\to$ Float
    \item[] distance: Escale $\times$ Escale $\to$ Float
    \item[] setBestTime: Escale $\times$ Float $\to$ Escale
    \item[] getBestTime: Escale $\to$ Float
\end{itemize}

\textbf{Préconditions:}
\begin{itemize}
    \item[] $\forall e1 \in \text{Escale}, \forall t \in \text{Float}$
    \item[] $\forall t, t \geq 0$, setBestTime(e1, t)
\end{itemize}

\textbf{Axiomes:}
\begin{itemize}
    \item[] $\forall e1, e2 \in \text{Escale}, \forall x, y \in \text{Float}, \forall n \in \text{String}, \forall t \in \text{Float}_{\geq0}$
    \item[] getX(create(n, x, y)) = x
    \item[] getY(create(n, x, y)) = y
    \item[] getName(create(n, x, y)) = n
    \item[] distance(e1, e2) = $\sqrt{(getX(e2) - getX(e1))^2 + (getY(e2) - getY(e1))^2}$
    \item[] getBestTime(setBestTime(e1, t)) = t
\end{itemize}

\subsection{Course}

\textbf{Type:}
\begin{itemize}
    \item[] Course
\end{itemize}

\textbf{Utilise:}
\begin{itemize}
    \item[] Escale, Integer, Float, Boolean
\end{itemize}

\textbf{Opérations:}
\begin{itemize}
    \item[] create: Escale $\times$ Escale $\to$ Course
    \item[] isCircle: Course $\to$ Boolean
    \item[] getEscalesNumber: Course $\to$ Integer
    \item[] getEtapesNumber: Course $\to$ Integer
    \item[] getBestToatalTime: Course $\to$ Float
    \item[] getBestTime: Course $\times$ Integer $\to$ Float
    \item[] add: Course $\times$ Escale $\to$ Course
    \item[] remove: Course $\to$ Course
\end{itemize}

\textbf{Préconditions:}
\begin{itemize}
    \item[] $\forall e1, e2 \in \text{Escale}$
    \item[] $\forall e1, e2, e1 \neq e2 \land getBestTime(e1) = 0$, create(e1, e2)
\end{itemize}

\textbf{Axiomes:}
\begin{itemize}
    \item[] $\forall c \in \text{Course}, \forall e1, e2 \in \text{Escale}, \forall i \in \text{Integer}, \forall t \in \text{Float}_{\geq0}$
    \item[] isCircle(create(e1, e2)) = False
    \item[] isCircle(add(create(e1, e2), e1)) = True
    \item[]
    \item[] getEscalesNumber(create(e1, e2)) = 2
    \item[] getEscalesNumber(add(c, e1)) = getEscalesNumber(c) + 1
    \item[]
    \item[] getEtapesNumber(create(e1, e2)) = 1
    \item[] getEtapesNumber(add(c, e1)) = getEtapesNumber(c) + 1
    \item[]
    \item[] getBestTotalTime(create(e1, e2)) = getBestTime(e1) + getBestTime(e2)
    \item[] getBestTotalTime(add(c, e1)) = getBestTotalTime(c) + getBestTime(e1)
    \item[]
    \item[] getBestTime(create(e1, e2), 0) = getBestTime(e1)
    \item[] getBestTime(create(e1, e2), 1) = getBestTime(e2)
    \item[]
    \item[] remove(add(c, e1)) = c
\end{itemize}


\section{Question(s)}
%%%%%%%%%%%%%%%%%%%%%%%

%%%%%%%%%%%%%%%%%%%%%%%%%%%%%%%%%%%%%%%%%%%%%%%%%%%%%%%%%%%%%%%%%%%%%%%%%%%%%%%%%%%%%%%%%%
% Dans cette section, vous pouvez rédiger les questions que vous avez sur le projet ou   %
% formuler ce qui reste incompris pour vous.                                             %
%                                                                                        %
% Nous en discuterons lors de la rencontre feedback sur votre production pour le         %
% milestone                                                                              %
%%%%%%%%%%%%%%%%%%%%%%%%%%%%%%%%%%%%%%%%%%%%%%%%%%%%%%%%%%%%%%%%%%%%%%%%%%%%%%%%%%%%%%%%%%

\end{document}
