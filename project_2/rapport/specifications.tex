% !TEX root = ./main.tex
%%%%%%%%%%%%%%%%%%%%%%%%%%%%%%%%%%%%%%%%%%%%%%%%%%%%%%%%%%%%%%%%%%%%%%%%%%%%%%%%%%%%%%%%%%
% Dans cette section, spécifiez formellement chacune des fonctionalités.                 %
%%%%%%%%%%%%%%%%%%%%%%%%%%%%%%%%%%%%%%%%%%%%%%%%%%%%%%%%%%%%%%%%%%%%%%%%%%%%%%%%%%%%%%%%%%
\section{Specifications}\label{specifications}
%%%%%%%%%%%%%%%%%%%%%%%%

\subsection{Constructeur}


\[
\begin{aligned}
&\text{course\_create\_list}: \text{Escale} \times \text{Escale} \to \text{Course} \\
&\text{pre}: e_1 \neq e_2 \land e_1 \neq \text{NULL} \land e_2 \neq \text{NULL} \land \text{escale\_time}(e_1) = 0 \\
&\text{post}: \\
&\quad \text{taille course à la création} = 2 \\
&\quad \text{course}\rightarrow\text{escale} = e1 \\
&\quad (\text{course}\rightarrow\text{next})\rightarrow\text{escale} = e2 \\
\end{aligned}
\]



\[
\begin{aligned}
&\text{course\_create\_array}: \text{Escale} \times \text{Escale} \to \text{Course} \\
&\text{pre}: e_1 \neq e_2 \land e_1 \neq \text{NULL} \land e_2 \neq \text{NULL} \land \text{escale\_time}(e_1) = 0 \\
&\text{post}: \\
&\quad \text{escale\_count} = 2 \\
&\quad \text{escales[0]} = e_1 \\
&\quad \text{escales[1]} = e_2 \\
\end{aligned}
\]


\[
\begin{aligned}
&\text{escale\_create}: \text{String} \times \text{Float} \times \text{Float} \to \text{Escale} \\
&\text{pre}: \text{name} \neq \text{NULL} \\
&\text{post}: \\
&\quad \text{escale}\rightarrow\text{name} = \text{name} \\
&\quad \text{escale}\rightarrow\text{x} = x \\
&\quad \text{escale}\rightarrow\text{y}= y \\
&\quad \text{escale\_get\_best\_time}(e) = 0
\end{aligned}
\]


\subsection{Transformateur}

\[
\begin{aligned}
&\text{course\_pop\_list}: \text{Course} \to \text{Course} \\
&\text{pre}: course \neq \text{NULL} \land \text{course\_escale\_count(course)} > 0\\
&\text{post}: \\
&\quad \text{Si course\_escale\_count(course)} = 1, \text{alors } \text{return NULL} \text{ et la mémoire est libérée} \\
&\quad \text{Sinon, } \text{supprimer le dernier escale et libérée} \\
\end{aligned}
\]

\[
\begin{aligned}
&\text{course\_pop\_array}: \text{Course} \to \text{Course} \\
&\text{pre}: course \neq \text{NULL} \land \text{course\_escale\_count(course)} > 0 \\
&\text{post}: \\
&\quad \text{supprimer le dernier escale (escales[escales\_count - 1]) et libérée} \\
&\quad \text{escales\_count} = \text{escales\_count} - 1 \\
\end{aligned}
\]


\[
\begin{aligned}
&\text{course\_append\_array}: \text{Course} \times \text{Escale} \to \text{Course} \\
&\text{pre}: course \neq \text{NULL} \land e \neq \text{NULL} \\
&\text{post}: \\
&\quad \text{Si } \text{escales\_size} == \text{escales\_count}, \text{alors escales est réalloué dynamiquement et} \\
&\quad \text{escales\_size} = \text{escales\_size} * 2 \\
&\quad \text{escales[escales\_count]} = e \\
&\quad \text{escales\_count} = \text{escales\_count} + 1
\end{aligned}
\]


\[
\begin{aligned}
&\text{course\_append\_list}: \text{Course} \times \text{Escale} \to \text{Course} \\
&\text{pre}: e \neq \text{NULL} \\
&\text{post}: \\
&\quad \text{Si } course = \text{NULL}, \text{alors une nouvelle course est créée et contient e} \\
&\quad \text{Sinon, e est ajoutée à la fin de la liste chaînée} \\
\end{aligned}
\]

\subsection{Observateur}



\[
\begin{aligned}
&\text{course\_is\_circuit}: \text{Course} \to \text{Boolean} \\
&\text{pre}: course \neq \text{NULL} \\
&\text{post}: \\
&\quad \text{si } \text{première étape} == \text{dernière étape return: true} \\
&\quad \text{sinon return: false}
\end{aligned}
\]



\[
\begin{aligned}
&\text{course\_best\_time\_at\_array}: \text{Course} \times \text{Integer} \to \text{Float} \\
&\text{pre}: course \neq \text{NULL} \land 0 \leq \text{index} < \text{course\_escale\_count(course)} \\
&\text{post}: \\
&\quad \text{return: escale\_get\_best\_time(escales[index])}
\end{aligned}
\]

\[
\begin{aligned}
&\text{course\_best\_time\_at\_list}: \text{Course} \times \text{Integer} \to \text{Float} \\
&\text{pre}: course \neq \text{NULL} \land 0 \leq \text{index} < \text{course\_escale\_count(course)} \\
&\text{post}: \\
&\quad \text{on cherche un escal, de manière récursive}\\
&\quad \text{return: escale\_get\_best\_time(escale)}
\end{aligned}
\]


\[
\begin{aligned}
&\text{escale\_get\_x}: \text{Escale} \to \text{Float} \\
&\text{pre}: escale \neq \text{NULL} \\
&\text{post}: \\
&\quad \text{return: escale} \rightarrow x
\end{aligned}
\]


\[
\begin{aligned}
&\text{escale\_get\_y}: \text{Escale} \to \text{Float} \\
&\text{pre}: escale \neq \text{NULL} \\
&\text{post}: \\
&\quad \text{return: escale}\rightarrow y
\end{aligned}
\]



\[
\begin{aligned}
&\text{escale\_distance}: \text{Escale} \times \text{Escale} \to \text{Float} \\
&\text{pre}: e_1 \neq \text{NULL} \land e_2 \neq \text{NULL} \\
&\text{post}: \\
&d = 2R \arcsin \left( \sqrt{\sin^2 \left(\frac{\Delta \phi}{2}\right) + \cos(\phi_1) \cos(\phi_2) \sin^2 \left(\frac{\Delta \lambda}{2}\right)} \right)
\end{aligned}
\]



